\documentclass[12pt,a4paper,openright,oneside]{article}
\usepackage{amsfonts, amsmath, amssymb,latexsym,amsthm, mathrsfs, enumerate}
\usepackage[spanish]{babel}
\usepackage{epsfig}

\parskip=5pt
\parindent=15pt
\usepackage[margin=1.2in]{geometry}
\usepackage{graphicx}
\usepackage{listings}
\usepackage[latin1]{inputenc}

\setcounter{page}{0}


\numberwithin{equation}{section}
\newtheorem{teo}{Teorema}[subsubsection]
\newtheorem*{teo*}{Teorema}
\newtheorem*{prop*}{Proposici�}
\newtheorem*{corol*}{Corol�lari}
\newtheorem{prop}{Proposici�}[subsubsection]
\newtheorem{corol}{Corol�lari}[subsubsection]
\newtheorem{lema}{Lema}[subsubsection]
\newtheorem{defi}{Definici�}[subsubsection]
\newtheorem{nota}{Notaci�}

\theoremstyle{definition}
\newtheorem{prob}{Problema}
\newtheorem*{sol}{Soluci�}
\newtheorem{ex}{Exemple}
\newtheorem{exs}{Exemples}
\newtheorem{obs}{Observaci�}
\newtheorem{obss}{Observacions}

\def\qed{\hfill $\square$}

\renewcommand{\refname}{Bibliografia}
% --------------------------------------------------
\usepackage{fancyhdr}

\lhead{}
\lfoot{}
\rhead{}
\cfoot{}
\rfoot{\thepage}

\begin{document}

\bibstyle{plain}

\thispagestyle{empty}

\begin{titlepage}
\begin{center}
\begin{figure}[htb]
\begin{center}
\includegraphics[width=6cm]{ub.png}
\end{center}
\end{figure}

\textbf{\LARGE Trabajo final de grado} \\
\vspace*{.5cm}
\textbf{\LARGE GRADO DE INFORM�TICA } \\
\vspace*{.5cm}
\textbf{\LARGE Facultat de Matem\`atiques \\ Universitat de Barcelona} \\
\vspace*{1.5cm}
\rule{16cm}{0.1mm}\\
\begin{Huge}
\textbf{Visualizaci�n de redes inteligentes} \\
\end{Huge}
\rule{16cm}{0.1mm}\\

\vspace{1cm}

\begin{flushright}
\textbf{\LARGE Autor: Martin Azpillaga Aldalur}

\vspace*{2cm}

\renewcommand{\arraystretch}{1.5}
\begin{tabular}{ll}
\textbf{\Large Director:} & \textbf{\Large Dr. Jes�s Cerquides } \\
\textbf{\Large Realitzat a:} & \textbf{\Large  IIIA   } \\
 & \textbf{\Large (nom del departament)} \\
\\
\textbf{\Large Barcelona,} & \textbf{\Large \today }
\end{tabular}

\end{flushright}

\end{center}










\end{titlepage}


\newpage
\pagenumbering{roman} 

\section*{Abstract}

Example abstract.

\section*{Res�men}
Res�men

\newpage 

\tableofcontents

\newpage

\pagenumbering{arabic} 
\setcounter{page}{1}
\section{El entorno(1)}
\subsection{La humanidad necesita energ�a el�ctrica}
Hay gran inter�s por parte de goviernos en invertir en investigar maneras m�s eficientes de tratar con la energ�a.
\subsection{La idea de los Smart Grids}
Auge de energ�as renovables personales. Mencionar art�culos donde se tratan smart grids.
\subsection{Problemas: �Como se tradea la energ�a?}
Quien controla todo el flujo, cuando se hacen los intercambios, como se hacen los intercambios, como pueden interferir los usuarios en estos intercambios. Mencionar reglamentos de otros pa�ses y estado actual de Espa�a.

\newpage

\section{Definiendo el problema(2)}

\subsection{El problema}
Explicar qu� se intenta resolver
\subsection{Partes del problema: Creaci�n de grafo, CEAP, Simulaci�n, Visualizaci�n, Reports}
Explicar por que existe cada parte y por que est� diferenciado del resto.
\subsection{Buscando tecnolog�as adecuadas}
Explicar programas adeuados para cada parte as� como posibles distintas maneras de implementar cada aspecto a grandes rasgos
\subsection{A partir de ahora}
Iremos explicando cada parte desde el n�cleo (CEAP) hasta el exterior (Visualizaci�n)
\newpage

\section{CEAP(5)}
\subsection{Se me present� este proyecto.} 
\subsection{Formalizaci�n en participant, ILV, PLV, Link.} 
\subsection{RadPro. }
\subsection{Limitaciones y ampliaciones posibles.}
\section{Simulaci�n(8)}
\subsection{Filosof�a de java. POO, encapsulaci�n, interfaces.}
\subsection{Input/Output. }
\subsection{Diagramas. }
\subsection{Modelado de una casa}
\subsection{Battery, }
\subsection{Generator,} 
\subsection{Appliance, }
\subsection{Bid.}
\section{Visualizaci�n(5)}
\subsection{Filosof�a de Unity. Componentes/no.} 
\subsection{Men� din�mico.} 
\subsection{Animaciones.}
\section{Reports(2)}
\subsection{A�n por decidir.}
\section{Creaci�n de la ciudad(5)}
\subsection{Algoritmos: Steiner tree, algoritmo RTT, algoritmo nearest.}
\subsection{ Ampliar a Open Street Map, CityEngine.}

\section{Conclusiones(1)}

adf.
\normalfont

\newpage

\begin{thebibliography}{25}
\bibitem{pari} Batut, C.; Belabas, K.; Bernardi, D.; Cohen, H.; Olivier, M.: User's guide to \textit{PARI-GP},  \newline \texttt{pari.math.u-bordeaux.fr/pub/pari/manuals/2.3.3/users.pdf}, 2000.
\end{thebibliography}
\end{document} 

